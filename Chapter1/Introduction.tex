\chapter{Introduction}
\label{chapterlabel1}
\epigraph{``Begin at the beginning," the King said gravely, ``and go on till you come to the end: then stop."}{--- \textup{Lewis Carroll}, Alice in Wonderland}


Simulation systems of \ac{mri} experiments are essential tools for a wide variety of research directions.
For example, in order to detect and eliminate artifacts, one can use an \ac{mri} simulator to differentiate between hardware imperfections and the effects of MR physics,
while in the field of correction algorithms, one can simulate \ac{mri} data sets that can provide the needed ground truth.
In general, validating and interpreting experimental results benefits from comparisons to simulated data.
Moreover, an \ac{mri} simulation system can provide an educational platform for future radiologists.

\hfill

Existing \ac{mri} simulation systems use different approaches to modelling the \ac{mri} experience and thus differ in closeness to reality, accuracy, extensibility towards more complex phenomena and computational burden.
In general, there are four main categories in which \ac{mri} simulators can be encapsulated based on how they solve the Bloch equations.

\hfill

The first category comprises of simulators based on the steady-state solutions of the Bloch equation of known pulse sequences \cite{Bobman1985} \cite{Lufkin1986} \cite{Ortendahl1984} \cite{Riederer1984}.
In these simulators, $T_1$, $T_2$ and proton density maps are used together with signal equations 
%of image intensity 
to produce MR images.
However, these simulators are simplistic, are used mainly for educational purposes, cannot simulate the process of MR acquisition and are not able to simulate image artifacts such as: chemical shift, intra-voxel dephasing, slice selection imperfections, aliasing and susceptibility artifacts.

\hfill

A second category of simulators is based on the k-space formalism \cite{Petersson1993}.
These simulators use the inverse Fourier transform to generate the ideal MR signal from a user-defined spin density image.
Then, using k-space trajectories of known MR sequences, the ideal amplitudes are sampled to mimic the process of MR acquisition and are corrected to simulate relaxation phenomena.
However, these types of simulations are limited to operations that influence the MR signal in a mathematical linear way.
This means that more complex scenarios such as non-linear gradients or relaxation during the application of an RF pulse cannot be incorporated in this framework.

\hfill

A third approach is based on analytical time-discretised solutions of the Bloch equations \cite{Bittoun1984} \cite{Drobnjak2006} \cite{Jochimsen2006} \cite{Kwan1999} \cite{Liu2017} \cite{Yoder2004}.
This approach is closer to reality than the previous ones, but it is limited in terms of more complex scenarios such as simultaneously having RF excitation and time-varying gradient fields, where no general analytical solution exists.
%, such as simultaneously having RF excitation and time-varying gradient fields.

\hfill

Finally, the fourth category consists of simulators which numerically solve the Bloch equations \cite{Olsson1995} \cite{Summers1986} \cite{Stocker2010}.
In the past, these simulators have been limited to small isochromat systems and simple MR sequences due to the lack of computational power of the available systems at the time \cite{Olsson1995} \cite{Summers1986}.
Today, these simulation systems benefit from the use of dedicated ordinary differential equation solvers and high performance computing systems
%which can increase the complexity of these simulations
\cite{Stocker2010}.
%\cite{Fortin2016}.
However, when coupling these systems with more complex MRI scenarios such as rapid multi-pulse gradient spoiled sequences, these systems end up becoming computationally very slow.

% \hfill

% Bloch-based numerical simulations of an \ac{mri} experiment are, in their most general form, a demanding task.
% This can be attributed to the fact that a large collection of isochromats needs to be simulated to obtain realistic results \cite{Shkarin1997}. 
% For this reason, one preferred approach is to create an \ac{mri} simulation system based on simplified analytical signal expressions \cite{Hacklander2005} \cite{Torheim1994} \cite{Yoder2004}.
% However, this approach is limited when it comes to modelling more complex artefacts 
% %such as time-varying gradient fields 
% where no general analytical solution exists.
% A different approach is then based on numerical modelling of the \ac{mri} processes \cite{Summers1986} \cite{Olsson1995}.
% At first, this has been limited to simulations of small spin systems with reduced flexibility due to existing computer hardware and software architectures.
% More recently, with the growing availability of high-performance computing, \ac{mri} simulation systems have become increasingly more complex, allowing for simulating various system imperfections \cite{Benoit-Cattin2005} \cite{Drobnjak2006}, while also providing dedicated graphical interfaces for pulse sequence design \cite{Stocker2010}.
% Overall, current \ac{mri} software systems can provide accurate simulations of MR imaging processes in reasonable time-scales, but several limitations still exist.

\hfill

The \ac{mri} simulators that are freely available today to be downloaded and used are based on the last two types of approaches.
These systems allow for simulations of various system imperfections \cite{Drobnjak2006}, 
provide dedicated graphical interfaces for pulse sequence design \cite{Jochimsen2004} \cite{Overall2007} \cite{Stocker2010} 
or
employ a generalized tissue model with multiple exchanging proton pools \cite{Liu2017}.
However, current \ac{mri} software systems lack support for a number of key features that prevent the simulation of advanced \ac{mri} sequences with the realism required.
These limitations are presented in the following subsection.

% Overall, current \ac{mri} software systems can provide accurate simulations of some MR imaging processes in reasonable time-scales.

% \hfill

% The major limitation of existing \ac{mri} simulation systems comes from their inability to accurately simulate fast multi-pulse gradient spoiled sequences.
% More specifically, in these types of sequences it is difficult to accurately reproduce NMR signals for continuous objects using a finite number of isochromats.
% In other words, a large intra-voxel isochromat collection is needed to produce reliable signals and to not allow for spurious echoes \cite{Benoit-Cattin2005} \cite{Kwan1999}.
% Hence, currently available simulation systems have the following problems:
% \begin{itemize}
%     \item They do not simulate voxel isochromat distributions and consequently cannot accurately simulate the corresponding signals \cite{Drobnjak2006} \cite{Overall2007}.
    
%     \item They do not benefit from high enough processing speeds to allow these simulations to be performed within reasonable time scales \cite{Drobnjak2006} \cite{Jochimsen2006} \cite{Stocker2010} \cite{Overall2007}.
    
%     \item They treat the effect of spoiler gradients on the magnetisation vectors as a zeroing event \cite{Liu2017}
% \end{itemize}

% do not simulate voxel isochromat distributions and consequently cannot accurately simulate the corresponding signals \cite{Drobnjak2006} \cite{Overall2007}, or they do not benefit from high enough processing speeds to allow these simulations to be performed within reasonable time scales \cite{Jochimsen2006}.

% For example, one recently developed quantitative \ac{mri} approach called \ac{mrf} requires an accurate modelling of the spoiling gradient effects on the intra-voxel transverse magnetisation.
% This can be achieved through fine discretisation of the digital input object but, with current approaches, it cannot be performed within reasonable time scales.
% Moreover, by adding tissue motion or system imperfections to the model, the complexity of the simulation increases even further.

\hfill

% The major limitation of existing MRI simulation systems comes from their inabil-ity to simulate more advanced MRI experiments, while also modelling a large col-lection of spins. For example, one recently developed quantitative MRI approachcalled Magnetic Resonance Fingerprinting (MRF) requires an accurate modellingof the spoiling gradient effects on the intra-voxel transverse magnetisation. Thiscan be achieved through fine discretisation of the digital input object but, with current approaches, it cannot be performed within reasonable time scales. Moreover,by adding tissue motion or system imperfections to the model, the complexity ofthe simulation increases even further.

% \ac{mri} is a non-invasive imaging technique based on the physical phenomena of nuclear magnetic resonance (NMR).
% \ac{mri} has many applications in the biomedical sciences such as the study of anatomy, pathology and even function.
% In a clinical environment, \ac{mri} is often preferred over other imaging modalities as it provides several unique advantages.
% Among these advantages, \ac{mri} offers high resolution images, very good soft tissue contrast, and does not use ionizing radiation.
% In addition, it can provide images with arbitrary orientation, as well as true three dimensional images.
% Unlike other imaging modalities, the \ac{mri} process can be sensitised to a wide range of morphological and physiological parameters, such as flow, perfusion, blood oxygenation, and many others.

% \hfill

% \ac{mri} has a few potential drawbacks. 
% First, MR images are generally qualitative, where the contrast between different tissue types is the primary source of information for characterizing anatomy or pathology. 
% However, the MR research community is determined to overcome this limitation by developing pure quantitative approaches which can measure tissue properties.
% One recently developed quantitative \ac{mri} approach called \ac{mrf} has shown great potential in simultaneously quantifying a range of different tissue properties in one acquisition.
% Secondly, the MR imaging process is sensitive to movement.
% This is because the time required for the majority of MR sequences to collect the data necessary to produce an image is longer than most types of physiological motion, such as involuntary patient motion, cardiac motion, respiratory motion, vessel pulsation, pulsatile flow and cerebrospinal fluid flow.
% This limitation becomes extremely important in quantitative approaches where both accuracy and precision are needed.

% \hfill

% One way to investigate how motion affects an MR scan is by doing experimental work.
% This, however, is time consuming and, more importantly, often hard to control the precise nature of the motion being tested.
% The alternative is to simulate the entire \ac{mri} pipeline together with a motion model. 
% The advantage over experimental work is that simulation provides a controlled way of studying how motion impacts the final images.
% More importantly, in the case of an MR Fingerprinting experiment, a realistic simulation framework provides a way of understanding how different types of motion affect the quantification of tissue properties.

% % % % % % % % % % % % % % % % 
\section{Statement of Problem}\label{chapterlabel1sec1}
There is a need to have a realistic \ac{mri} simulation framework to accurately and precisely investigate advanced MR experiments.

\hfill

% \subsection{Simulation Systems}
% A simulation system could model the full MR acquisition process producing realistic images from a computational representation of a tissue or phantom and the MR scanning parameters.
% An \ac{mri} simulator allows full control over the simulation parameters thus enabling the understanding of how different artefacts will affect the MR images.
% Moreover, an \ac{mri} simulation framework can be further used to investigate how these artefacts impact the quantification of tissue properties in more advanced MR sequences such as magnetic resonance fingerprinting.

% % % % % % % % % % %
% \subsection{Limitations of existing simulation systems}
\large \textbf{Limitations of existing simulation systems} \normalsize

Current MR simulation systems 
%lack support for a number of key features that prevent the simulation of more advanced \ac{mri} sequences with the realism required. 
%More specifically, existing systems 
exhibit at least one of the following problems:

\begin{enumerate}

    \item \textit{There is little support for accurate gradient spoiling.}
    Gradient spoilers are commonly used in MR sequences to eliminate or attenuate residual transverse magnetisation.
    Gradient spoilers are treated differently by different currently available MRI simulation systems.
    %At the end of a pulse sequence or a preparatory RF pulse, residual trans- verse magnetization can remain. The residual transverse magnetization, if not eliminated, can produce a spurious signal that interferes with the desired signal in subsequent data acquisition, causing image artifacts (Figure 10.25a) (Haase et al. 1986; Haacke et al. 1991)
    Some available simulators calculate specific intra-voxel isochromat dephasing \cite{Drobnjak2006} \cite{Jochimsen2006}.
    This, however, does not account for rapid multi-pulse gradient-echo sequences where intra-voxel rephasing can occur later in the sequence.
    Other systems null the transversal magnetization automatically when a spoiling gradient is applied \cite{Benoit-Cattin2005} \cite{Liu2017}.
    Finally, other systems can simulate a large number of isochromats to calculate the effect of a gradient spoiler, but require unrealistically long times to produce images \cite{Stocker2010}.
    
    % This type of rapid multi-pulse sequence is the current state-of-the-art in \ac{mrf} acquisitions.
    % However, in currently available simulation systems the spoiling gradient associated with this sequence is either treated as a zeroing event or, if the object is discretised finely enough, such simulations take an unrealistically long time to produce images.
    % Therefore, there are currently no simulation systems that offer an out-of-the-box solution for this type of sequence.
    
    % \item Accurate simulations of fast multi-pulse steady-state sequences with gradient spoiling are currently not feasible. \\
    % Due to their inherent need for a very large number of isochromats, the current systems which allow for such simulations would take an unrealistically long time to produce images.
    
    \item \textit{There is little support for very fast simulations through the use of a \ac{gpu} architecture.}
    Current technological advancements in the computer architecture community allows for very fast computational approaches through the use of massively parallel computer hardware.
    However, most MR simulators rely on \ac{cpu} parallelisation techniques 
    \cite{Benoit-Cattin2005}
    \cite{Drobnjak2006} \cite{Stocker2010}.
    Although faster than sequential approaches, \ac{cpu} based techniques are still a few orders of magnitude slower than \ac{gpu}s.
    This, coupled with the need for an accurate simulation of a gradient spoiling event, shows that the current systems are not well suited for finely discretised digital objects. \\
    %(mrilab is cuda 2.0, more recently is 9.0; also not as customisable as you'd like)

%     \hfill

% 	Moreover, in the case of \ac{mrf} simulations, the following major limitation exists: \\

%     \item \textit{There is currently no available simulation system that models the full process of \ac{mri} acquisition during an \ac{mrf} experiment.}
%     MR acquisition relies on a series of magnetic fields being applied over a sample in order to spatially encode it into Fourier space.
%     After this step, a reconstruction algorithm is used to retrieve an image representing the object.
%     However, many of the artefacts known to the process of MR can happen during acquisition.
%     The failure to faithfully model this process can prevent the realistic investigation of many commonly occurring artefacts.
\end{enumerate}

% % % % % % % % % % % % % % % % 
\section{PhD Contribution}\label{chapterlabel1sec2}

The main aim of my PhD project is to develop a magnetic resonance imaging simulation framework capable of producing \ac{mri} datasets in an accurate and efficient manner.
Moreover, I aim to use this \ac{mri} simulator to investigate the impact motion 
has on the quantitative maps produced by a magnetic resonance fingerprinting protocol.
%in a novel quantitative \ac{mri} protocol known as magnetic resonance fingerprinting.
I will achieve this by making the following contributions:

%The main aim of my PhD project is to develop a magnetic resonance imaging simulation framework and to demonstrate its application on a novel quantitative \ac{mri} protocol known as \ac{mrf}.
%Towards this goal, I have developed an initial proof-of-concept image space simulation pipeline of an \ac{mrf} experiment.
%Moreover, I have investigated the impact of in-plane rigid motion applied at various time points during the scan on the final quantitative maps.
%This chapter is therefore focused on describing the relevant methodology for implementing the proposed framework and on presenting the key results and conclusions following this investigation.

\begin{enumerate}
    % 	\item \textit{Upgrade an existing simulator to enable the simulation of advanced Magnetic Resonance Imaging experiments.}
    % POSSUM \cite{Drobnjak2006} is a state-of-the-art open source \ac{mri} simulation tool that allows for rapid and accurate reproductions of magnetic resonance imaging datasets.
    % 	Recently, is has been successfully extended to reproduce realistic diffusion-weighted datasets \cite{Graham2016}.
    % 	By upgrading it to allow for \ac{gpu} parallelisation and an easily extensible programming interface, POSSUM will be able to realistically perform simulations of more advanced \ac{mri} acquisitions.
    % 	Moreover, this simulator will become freely available for anyone to use and design their own experiments. 
    % 	This contribution tackles limitations 1 and 2.
    
	\item \textit{Develop a simulation system for advanced \ac{mri} sequences.}
	My aim is to develop an open source \ac{mri} simulation tool that allows for rapid and accurate reproductions of magnetic resonance imaging datasets.
	The computational complexity required in simulating more advanced sequences and realistic digital objects will be addressed
    through \ac{gpu} parallelisation and the newest programming standards offered by CUDA\footnote{CUDA is NVIDIA's parallel computing architecture that enables dramatic increases in computing performance by harnessing the power of the GPU} 9.x.
    This contribution tackles limitations 1 and 2.
	
	\item \textit{Demonstrate its application on a novel quantitative \ac{mri} protocol known as \ac{mrf}.}
	To demonstrate the potential value of the newly developed simulation system, I will apply it to model the full process of \ac{mri} acquisition during an \ac{mrf} experiment.
% 	This contribution tackles limitation 3.
	
    % 	Starting with 2013, \ac{mrf} has caught the attention of many MR enthusiasts who aim to make this imaging modality quantitative.
    % 	This technique has shown a lot of promise in providing simultaneous measurements of multiple parameters of interest, such as the tissue relaxation times, the spin density and others, using a single acquisition protocol.
    % 	As it is slowly paving its way into the radiology departments, it has become timely to allow for accurate and reliable simulations of the sequence, thus enabling the \ac{mrf} community to further improve it.
    %   This contribution tackles limitation 3.

	\item \textit{Investigate the impact of motion on the quantitative maps produced by an \ac{mrf} experiment.}
	Motion during a magnetic resonance imaging scan is one of the most frequent sources of artefacts in \ac{mri} \cite{Zaitsev2015}. 
	This is because the time required for the majority of MR sequences to collect the signal is longer than most types of physiological movements.
	This limitation of \ac{mri}, coupled with the need for accuracy and precision in quantitative approaches, makes this investigation not just timely, but also needed.

\end{enumerate}

% % % % % % % % % % % % % % % % 
\section{Thesis Outline}\label{chapterlabel1sec5}

My thesis consists of the following chapters:
\begin{itemize} 
	\item First, \textbf{\textit{Chapter \ref{chapterlabel1}}} introduces the problem statement and explains the motivation and importance of this PhD project.

	\item Second, \textbf{\textit{Chapter \ref{chapterlabel2}}} presents the current state-of-the-art in \ac{mri} and \ac{mrf} simulation systems, thus forming a comprehensive literature review.

	\item Third, \textbf{\textit{Chapter \ref{chapterlabel3}}} demonstrates an initial proof-of-concept image space simulation pipeline of an \ac{mrf} experiment.
	
	\item Fourth, \textbf{\textit{Chapter \ref{chapterlabel4}}} provides the overall conclusions for this piece of work and discusses the future work that needs to be done to finalise the PhD project.

	\item Finally, in \textbf{\textit{Appendix \ref{appendixlabelBackgroundMRI}}} and \textbf{\textit{Appendix \ref{appendixlabelBackgroundMRF}}} the relevant theoretical background for both \ac{mri} and \ac{mrf} are presented;
	\textbf{\textit{Appendix \ref{appendixlabelPhantom}}} contains a table with the $T_1$ and $T_2$ relaxation times used in the input object for our simulations;
	in \textbf{\textit{Appendix \ref{appendixlabelMotion}}} supplementary results can be found;
	while in \textbf{\textit{Appendix \ref{appendixlabel1}}} and \textbf{\textit{Appendix \ref{appendixlabel2}}} pseudocode for the \ac{mrf} dictionary generation is shown.

\end{itemize}









