
In this work I showed a proof-of-concept image space simulation pipeline of a magnetic resonance fingerprinting experiment.
The proposed method uses a physics-based approach to the MR acquisition process in order to simulate the MRI datasets.
Moreover, I investigated the impact of in-plane rigid motion applied at various time points during the scan on the final quantitative maps.
In this approach, the motion artifacts were introduced during the acquisition of the signal which made these simulations more realistic than other approaches where datasets were corrupted by applying geometric transformations in image space (see Section \ref{chapterlabel2sec2}).

\hfill

This investigation shows that MRF may not be as robust to motion as previously suggested and that the severity of its impact is strongly dependent on the type of motion, its onset and the dictionary construction.
However, motion is not necessarily restricted to in-plane movements.
Through-plane motion can happen during a real scan and it could have a different impact on the $T_1$ and $T_2$ maps.
Moreover, a real scan cannot use a fully sampled spiral readout due to hardware constraints, but uses a variable density spiral readout with multiple interleaves.
This is a limitation of my current approach, but I aim to cover it in future work.