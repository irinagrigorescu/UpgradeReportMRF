\chapter{Literature Review}
\label{chapterlabel2}
\epigraph{``Not everything that can be counted counts, and not everything that counts can
be counted."}{--- \textup{Albert Einstein}}

Magnetic Resonance Imaging simulation systems are an essential tool for a variety of research topics.
For example, in the field of pulse sequence optimisation, MRI simulations allow for the differentiation between hardware imperfections and the effects of MR physics, 
while in the field of correction algorithms, simulated data sets can provide the needed ground truth.
Moreover, controlled numerical MRI experiments can also be used for educational purposes.
This chapter presents an overview of both past and present MRI simulation systems.
The first section focuses on surveying the current literature in terms of MRI simulation, while the second section discusses the features of currently available MRI simulators.
Finally, the third section presents the more specific case of a Magnetic Resonance Fingerprinting experiment and how the current literature has simulated this novel sequence.

\hfill

% The aim of my PhD is to develop and apply a simulation system for Magnetic Resonance Imaging (MRI), capable of producing realistic datasets along with their artefacts.
% To enable an effective investigation of how commonly occurring MR artefacts affect these datasets, we focus our attention towards a novel quantitative MRI technique called Magnetic Resonance Fingerprinting (MRF).
% In order to achieve our goal, we begin with an investigation of existing MRI simulation systems, focusing on the current state-of-the-art.
% The first three sections are focused on general MR systems, while the last section discusses Magnetic Resonance Fingerprinting simulations.

%The first section of this chapter will provide a thorough review of existing MR simulation systems, focusing on the current state-of-the-art and a discussion of their limitations, while the second section will discuss current MRF simulations.
%, with a focus on how motion corrupted MRF reconstructions have been simulated.

% % % % % % % % Current MRI Simulation Systems
%\section{Magnetic Resonance Imaging Simulation Systems}
\section{MR Simulation: a general overview of past and current simulation systems}
\label{chapterlabel2sec1}
%: a review of past and recent simulation systems}

% Magnetic resonance imaging (MRI) is a non-invasive imaging technique based on the physical phenomena of nuclear magnetic resonance (NMR).
% MRI has many applications in the biomedical sciences such as the study of anatomy, pathology and even function.
% In a clinical environment, MRI is often preferred over other imaging modalities as it provides several unique advantages.
% Among these advantages, MRI offers high resolution images, very good soft tissue contrast, and does not use ionizing radiation.
% In addition, it can provide images with arbitrary orientation, as well as true three dimensional images.

% \hfill

% %MRI is an incredibly versatile imaging technique.
% %The MR signal can be sensitised to a wide range of morphological and physiological parameters, such as flow, perfusion, blood oxygenation, and many others.
% MRI has three distinct features that make it stand out among other diagnostic imaging modalities.
% First, MRI is an incredibly versatile imaging technique, allowing the MR signal to be sensitised to a wide range of morphological and physiological parameters.
% For this reason, MRI requires an increased level of understanding of the underlying physics to acquire and analyze the images.
% Second, the optimisation of an MRI pulse sequence is time consuming and does not exhaustively search the entire parameter space. 
% This process often involves phantoms, animals or human volunteers, and is based on an iterative process of parameter modification until the desired contrast and other image characteristics are found.
% Third, the quality of an MR image is dependent on the interaction between the patient and the hardware, or on the scanner hardware itself \cite{Graves2013}.
% It is therefore extremely important to understand the sources of image artifacts in order to eliminate them and avoid false diagnoses.

% \hfill

Numerical simulations of MR experiments have been around since the 1980s when Bittoun et al. published ``A computer algorithm for the simulation of any nuclear magnetic resonance (NMR) imaging method'' \cite{Bittoun1984}.
Following on their footsteps, many others have since then developed MRI simulation systems for a variety of reasons.
%One way to account for these important features and limitations is to build a software system capable of simulating the MRI process.
%Such simulators have been used for a number of purposes.
First, an MRI simulator that presents the same interface as a real scanner can be used as a training tool for physicists and clinicians.
Second, the parameter space can be exhaustively searched in a controlled way to create new sequences or to optimise existing ones.
And third, as MR imaging artifacts are often hard to avoid, a simulator could provide the ground truth for correction algorithms.
% In general, these software tools have been developed with a particular methodological question in mind.

\hfill

% %To date, a number of MRI physics simulators have been proposed for the reasons explained above.
%In order to develop an all purpose MRI physics simulator, few assumptions regarding the underlying physics should be made and all the main building blocks that make up a real MRI protocol should be integrated in the simulation system.
In its most general form, numerical simulation of an MRI experiment is a demanding task. 
This is due to the fact that many aspects of an MR experience need to be simulated in order to obtain realistic results.
From a high level perspective, such an experience can be described by a four stage pipeline.
First, \textbf{the object} that is being imaged (a phantom, an animal model or the body part of a human volunteer) is securely placed in the bore of the main magnet; second, \textbf{the pulse sequence} (the complete description of what the MR scanner should be doing) is being set up to activate the MR hardware; third, \textbf{the hardware} of the MRI scanner (the main magnet, gradient coils, radio frequency transmission and reception coils) acts upon the sample being imaged by exciting it with a collection of magnetic fields; and finally, \textbf{the reconstruction algorithm} (fast Fourier transform, parallel imaging reconstruction techniques, gridding, etc.) retrieves the final image.

\hfill

In this section, we focus on giving a brief description of how the current literature deals with simulating the MRI experience based on the four components described above.
%Next, we present the current state-of-the-art in MRI simulation systems as freely available software tools that are active today.
%Finally, we look at the more specific case of a Magnetic Resonance Fingerprinting experiment and how the current literature has simulated this novel sequence.

\hfill

% % % % % % % % % % % % % % % % % % % % % % % % 
% % % % % % % % % % % % % % % % % % % % % % % % 
% % % % % % % % % % % % % % % % % % % % % % % % 
\subsection{Object}

The object being imaged in an MRI scan or protocol is the first component of the MR experience pipeline.
%As the object needs to be simulated, 
There are three main criteria that are sufficient to fully describe it for the purpose of MRI simulation.
Firstly, the object needs to be \textbf{represented geometrically}. 
Secondly, the \textbf{tissue specific parameters} of interest to NMR need to be specified.
Lastly, the \textbf{change in position} that is ubiquitous to any real sample needs to be specified.
These three criteria are treated in different ways by different currently available simulators. 
It is the purpose of this section to present an overview of these differences.

% % % % % % % % % % % % % % % % % % % % % % % % 
\hfill

\large \textbf{Geometric Representation} \normalsize

There are many ways a three dimensional object can be represented in a computer, ranging from an unstructured set of 3D point samples (point cloud), to a connected set of 2D polygons (mesh), and even more complex representations such as hierarchical tree structures (octrees).
However, the most common object representation found in MRI simulation systems is a uniform grid of volumetric samples (voxels).
%The reason behind this is that most biomedical imaging systems such as MRI or CT use the same representation.
%The reason behind this is that this type of geometric representation is computationally efficient and therefore preferred in the MRI simulation world.

\hfill

This geometric representation has been adopted as early as 1984 when Bittoun et al. \cite{Bittoun1984} simulated one dimensional objects.
Following their steps, Stewart et al. \cite{Stewart1986} extended this approach to 2D objects, while Summers et al. \cite{Summers1986} and Olsson et al. \cite{Olsson1995} moved towards 3D objects.
The most recent MRI simulation systems by 
Yoder et al. \cite{Yoder2004}, 
Benoit-Cattin et al. \cite{Benoit-Cattin2005}, 
Baum et al. \cite{Baum2011}, 
Jochimsen et al. \cite{Jochimsen2004} \cite{Jochimsen2006}, 
Drobnjak et al. \cite{Drobnjak2006} \cite{Drobnjak2010}, 
Stocker et al. \cite{Stocker2010}, 
Xanthis et al. \cite{Xanthis2014} and 
Liu et al. \cite{Liu2013} \cite{Liu2014} \cite{Liu2016} use the same piecewise constant representation of the 3D input object.
%In all of these studies, the MR signal is generated by solving the Bloch equation at each point in the object.
%This approach becomes computationally demanding for high-resolution objects.

\hfill

%A more computationally efficient representation was presented in the simulation systems developed by 
A different representation was chosen by
Kwan et al. \cite{Kwan1997} \cite{Kwan1999}
where tissue templates \cite{Collins1995} were used instead.
These templates are three-dimensional anatomical images of distinct tissue types (e.g. one template of grey matter, another of white matter and a third of cerebro-spinal fluid (CSF)). 
However, this approach is restrictive in terms of object voxel properties as it does not allow for different object voxels to experience different susceptibility induced magnetic field inhomogeneities, so several artefacts due to rigid body motion or magnetic inhomogeneities cannot be modelled.

% % % % % % % % % % % % % % % % % % % % % % % % 
\hfill

\large \textbf{Tissue Specific Parameters} \normalsize

Tissue specific parameters refer to the set of chemical and physical characteristics of the object being imaged that are important to NMR.
These tissue properties influence the behaviour of the nuclei of interest which, in turn, affect the MRI signal.
In MR simulations, the parameters that are generally used are the relaxation times $T_1$ and $T_2$, the spin density $\rho$ and the magnetic susceptibility of different tissue types.
%In MRI, the signal depends on a range of chemical and physical characteristics of the object being imaged.

\hfill

The most common set of tissue specific parameters 
%used to characterize the input object that is 
found in the MRI simulation literature is composed of the proton density and the relaxation times.
%The image contrast in most MR imaging protocols used in clinics today is given by 
%For this reason, most MRI simulators use these three quantities to characterize each voxel in their object.
Both early simulator systems such as those created by
Bittoun et al. \cite{Bittoun1984},
Ortendahl et al. \cite{Ortendahl1984},
Riederer et al. \cite{Riederer1984},
Bobman et al. \cite{Bobman1985},
Lufkin et al. \cite{Lufkin1986},
Stewart et al. \cite{Stewart1986},
Summers et al. \cite{Summers1986},
Petersson et al. \cite{Petersson1993},
Torheim et al. \cite{Torheim1994}, 
Olsson et al. \cite{Olsson1995},
Brenner et al. \cite{Brenner1997},
Kwan et al. \cite{Kwan1997} and
more recent ones such as those created by
Yoder et al. \cite{Yoder2004},
Hacklander et al. \cite{Hacklander2005},
Benoit-Cattin et al. \cite{Benoit-Cattin2005},
Jochimsen et al. \cite{Jochimsen2004},
Overall et al. \cite{Overall2007},
Stocker et al. \cite{Stocker2010} and
Xanthis et al. \cite{Xanthis2014}
give values for the spin density, the longitudinal relaxation time $T_1$ and the transverse relaxation time $T_2$ to each element in the input object.

\hfill

% Similarly, 
%  use the same three tissue properties to define their object voxels.
However, Benoit-Cattin et al. \cite{Benoit-Cattin2005} 
and Stocker et al. \cite{Stocker2010} also introduce $T_2^*$ relaxation.
While the former computes this value by weighting the signal with the effect introduced by intra-voxel inhomogeneities, the latter simulates it by adding small Lorentzian distributed random off-resonance frequency terms to each simulated isochromat.
On the other hand, Liu et al. \cite{Liu2014} use both $T_2$ and $T_2^*$ maps for the input object, while 
Drobnjak et al. \cite{Drobnjak2006} use only $T_2^*$ values.
 
\hfill

Another important property is the magnetic susceptibility of different tissue types.
This property is related to the different molecular environments in which the nuclei can be found.
These different environments can shield the hydrogen protons from the full effects of an externally applied magnetic field, thus causing them to precess at different frequencies than expected.
This phenomenon causes the `chemical shift' artefact.
To date, MRI simulation systems developed by
Yoder et al. \cite{Yoder2004}, 
Drobnjak et al. \cite{Drobnjak2006}, 
Stocker et al. \cite{Stocker2010}, 
Kwan et al. \cite{Kwan1999}, 
Benoit-Cattin et al. \cite{Benoit-Cattin2005}, 
Xanthis et al. \cite{Xanthis2014} and 
Liu et al. \cite{Liu2013} 
have modelled this effect by 
introducing a frequency offset for a particular tissue type.

% \hfill

% Finally, interfaces of materials with different magnetic susceptibilities, such as tissue-air interfaces, can cause distortions in the main magnetic field.
% This phenomenon has been included in early simulators by 
% Bittoun et al. \cite{Bittoun1984}, and in more recent ones by
% Kwan et al. \cite{Kwan1999},
% Yoder et al. \cite{Yoder2004},
% Benoit-Cattin et al. \cite{Benoit-Cattin2005},
% Drobnjak et al. \cite{Drobnjak2010},
% Stocker et al. \cite{Stocker2010} and
% Xanthis et al. \cite{Xanthis2014}.

% % % % % % % % % % % % % % % % % % % % % % % % 
\hfill

\large \textbf{Motion} \normalsize

Motion is ubiquitous to any real life object and so the sample being imaged in an MRI experiment is never completely stationary.
Moreover, the time required for the majority of MR sequences to collect the necessary data is much longer than most types of physiological motion, including respiratory motion, vessel pulsation, CSF flow and even involuntary patient motion, which makes MRI particularly sensitive to this.

\hfill

Bulk motion is one important type of motion that can lead to slice misalignment, blurring of object edges, ghosting, loss of information or undesired strong signals \cite{Zaitsev2015}.
Drobnjak et al. \cite{Drobnjak2006}, Stocker et al. \cite{Stocker2010} and Jochimsen et al. \cite{Jochimsen2004} simulate rigid body motion at any time point during an MRI sequence, including during the acquisition process.
A different type of bulk motion that can be simulated is flow.
This type of motion is important in MRI because it can offer more subtle contrasts that can be used to image blood vessels and arteries.
Petterson et al \cite{Petersson1993} incorporate flow in their k-space formalism approach as a phase shift in the signal, while 
Fortin et al. \cite{Fortin2016} simulate arbitrarily complex fluid motion by extending an already existing simulator called JEMRIS \cite{Stocker2010}.

\hfill

Diffusion, or microscopic motion, is a more subtle type of movement.
This type of motion can be exploited through specialised pulse sequences to probe the underlying tissue microstructure.
While there is substantial work on simulating the diffusion signal arising from a single voxel (e.g. Camino \cite{Cook2006} and JEMRIS \cite{Stocker2010}),
full-brain images with diffusion contrast are more limited.
Nevertheless, a few examples exist.
Among them, Perrone et al. \cite{Perrone2016} produce realistic diffusion-weighted images, but they do not faithfully model the physics of MR acquisition,
while Graham et al. \cite{Graham2016} combine the simulation of the MR image acquisition process with a model-based representation of diffusion in order to produce realistic DW-MR images.

\hfill

Finally, a different type of miscroscopic motion involves the movement of spins between two or more macromolecular environments.
This type of motion can be used to generate MR images with more interesting contrasts.
For example, Chemical-Exchange Saturation-Transfer (CEST) techniques exploit this phenomenon to enable imaging certain compounds at concentrations that are too low to directly be detected through conventional imaging.
In terms of simulations, these phenomena require different molecular pools (free and bound) to be simulated, together with an exchange model.
This type of simulations are accounted for by Liu et al. \cite{Liu2017}.

\hfill

% % % % % % % % % % % % % % % % % % % % % % % % 
% % % % % % % % % % % % % % % % % % % % % % % % 
% % % % % % % % % % % % % % % % % % % % % % % % 
\subsection{Pulse Sequence}

The MRI pulse sequence is the second component of the MR experience pipeline.
A pulse sequence is a series of magnetic fields used in conjunction with data acquisition to produce MR images. 
By playing out the magnetic fields in a certain way, this `recipe' allows for differentiation of tissues in the images, or for sensitization of the MR signal to diffusion, flow or perfusion.
%Pulse sequences rely on a number of parameters to be set for the required contrast.
In the MRI simulation literature, there are two main approaches to how a pulse sequence can be defined.
Firstly, some MRI simulation systems offer a \textbf{predefined list of sequences} for the user to select from.
%Here, users are instructed to provide values for the most important sequence parameters such as: echo time, repetition period, inversion time and RF pulse flip angle.
Secondly, other simulation systems offer the users the possibility to \textbf{design their own sequences}.
%In this case users can either: provide an input file with a full description of the pulse sequence events at discretised time points, use the available tutorials and specialised application programming interfaces (API) to create their own sequences, or use the dedicated graphical user interface (GUI) module to develop pulse sequences.
Based on these two approaches, this section presents an overview of how different simulators treat the MRI pulse sequence.

% % % % % % % % % % % % % % % % % % % % % % % % 
\hfill

\large \textbf{Predefined Sequences} \normalsize

The most common readily available sequences in the MR simulation literature are the classic spin-echo and gradient-echo.
These types of sequences require that the user specifies two sequence specific parameters: $T_E$ (echo time) and $T_R$ (repetition time).
This is available in early MRI simulators developed by Ortendahl et al \cite{Ortendahl1984}, Riederer et al \cite{Riederer1984}, Bobman et al \cite{Bobman1985}, Lufkin et al \cite{Lufkin1986}, Torheim et al \cite{Torheim1994}, Simmons et al \cite{Simmons1996} and Hacklander et al \cite{Hacklander2005}.
Additionally, Ortendahl et al \cite{Ortendahl1984}, Torheim et al \cite{Torheim1994}, Simmons et al \cite{Simmons1996}, Torheim et al \cite{Torheim1994} and Hacklander et al \cite{Hacklander2005} include a third parameter, the $T_I$ (inversion time), to simulate inversion-recovery sequences.
Another important sequence parameter to be specified is the RF pulse flip angle. 
This feature is available in MRI simulators starting with 2005 \cite{Benoit-Cattin2005}.

\hfill

More complex sequences, such as \textit{PROPELLER-EPI}, \textit{TrueFISP}, \textit{multi-slice MDEFT} and \textit{fully flow-compensated FLASH} sequences, among others, are readily available through the simulation system developed by Jochimsen et al \cite{Jochimsen2004}.
As before, the user can input values for the most important sequence parameters such as: echo time, repetition period and RF pulse flip angle.


% % % % % % % % % % % % % % % % % % % % % % % % 
\hfill

\large \textbf{Design your own Sequences} \normalsize

Another approach to pulse sequence development is to allow the user to create their own.
Currently, there are three ways this can be accomplished.
Firstly, users are instructed to provide an input file with a full description of the pulse sequence events at discretised time points.
This approach is found in MRI simulators such as those created by Drobnjak et al \cite{Drobnjak2006} and Xanthis et al \cite{Xanthis2014}.
Secondly, users can take advantage of tutorials and specialised application programming interfaces (API) to create their own sequences.
This approach is found in the MRI simulation system developed by Jochimsen et al \cite{Jochimsen2004}, where an easily extensible programming class structure is provided.
Finally, users can benefit from the existence of GUI modules dedicated to pulse sequence development.
This can be found in simulators developed by Stocker et al. \cite{Stocker2010}, Xanthis et al \cite{Xanthis2014}, Jochimsen et al \cite{Jochimsen2004} and Liu et al \cite{Liu2013}.
%Moreover, Xanthis et al \cite{Xanthis2014}, Jochimsen et al \cite{Jochimsen2004} and Liu et al \cite{Liu2013} model more complicated RF pulse shapes such as sinc shaped, Gauss shaped, rectangular and adiabatic. 
%Finally, ODIN \cite{Jochimsen2004} also allows for composite pulse shapes by concatenating predefined ones.

\hfill

% % % % % % % % % % % % % % % % % % % % % % % % 
% % % % % % % % % % % % % % % % % % % % % % % % 
% % % % % % % % % % % % % % % % % % % % % % % % 
\subsection{Scanner Hardware}

The scanner hardware is the third component of the MR experience pipeline that needs to be simulated.
The main hardware components of a real MRI machine are: 
the main magnet bore which establishes the static magnetic field $B_0$, 
the gradient coils which cause controlled changes in the amplitude of the field,
the radiofrequency coils which are used to both receive and transmit magnetic fields,
the RF amplifier which is used to increase the power of the RF pulses,
the analog to digital converter which digitizes the received signal,
the RF shield which acts as a Faraday cage to trap the magnetic field lines inside the room, and
the computer console which is used by the investigator to view the results.

\hfill

For simulation purposes, three of these components can fully describe this step of the MRI experience pipeline.
These are: \textbf{the main magnet}, \textbf{the gradient coils} and \textbf{radiofrequency coils}.
These three components are treated differently by different currently available simulators.
It is the purpose of this section to present an overview of these differences.

% % % % % % % % % % % % % % % % % % % % % % % % 
\hfill

\large \textbf{Magnet} \normalsize

All MR scanners come equipped with a high field magnet.
In terms of field strength, the most widely available clinical scanners are the 1.5T and 3T fields, while in a research environment they can go up to 11.7T for humans and up to 23T for spectroscopy studies \cite{Morrow2000}.
In all cases, an important aspect of the static magnetic field is that it needs to be kept as homogeneous as possible through the imaged object.
For this, shim coils are used to adjust the homogeneity of the field \cite{Romeo1984}.

\hfill

From an MR simulation perspective, it is important to have the option of choosing the preferred field strength.
Indeed, early simulator systems such as those created by
Bittoun et al. \cite{Bittoun1984},
Ortendahl et al. \cite{Ortendahl1984},
Riederer et al. \cite{Riederer1984},
Bobman et al. \cite{Bobman1985},
Lufkin et al. \cite{Lufkin1986},
Stewart et al. \cite{Stewart1986},
Summers et al. \cite{Summers1986},
Petersson et al. \cite{Petersson1993},
Torheim et al. \cite{Torheim1994}, 
Olsson et al. \cite{Olsson1995},
Brenner et al. \cite{Brenner1997} and
Kwan et al. \cite{Kwan1997} allow for changing the static magnetic field strength $B_0$.
Similarly, more recent simulators such as those created by
Yoder et al. \cite{Yoder2004},
Hacklander et al. \cite{Hacklander2005},
Benoit-Cattin et al. \cite{Benoit-Cattin2005},
Jochimsen et al. \cite{Jochimsen2004},
Drobnjak et al. \cite{Drobnjak2006},
Overall et al. \cite{Overall2007},
Stocker et al. \cite{Stocker2010},
Liu et al. \cite{Liu2013} and 
Xanthis et al. \cite{Xanthis2014} allow for the same flexibility.

\hfill

Another important aspect of the main magnetic field is the presence of magnetic field inhomogeneities.
These types of distortions are simulated as deviations in the magnetic flux from the ideally homogeneous magnetic field.
In general, these main-field inhomogeneities are given as an arbitrary function of position, which can be specified either analytically or numerically.
Such inhomogeneities are simulated in 
Brenner et al. \cite{Brenner1997}, in
Liu et al. \cite{Liu2013} and in
Stocker et al. \cite{Stocker2010}.
%In addition, Drobnjak et al. \cite{Drobnjak2010} simulates the effects of time-varying magnetic fields by extending on their previous work with a model for changing magnetic fields at very high-resolution time-scales.

% % % % % % % % % % % % % % % % % % % % % % % % 
\hfill

\large \textbf{Gradient Coils} \normalsize

Gradient coils are part of the MR scanner hardware and are used to change the amplitude of the main magnetic field in a controlled way.
These spatial variations in the main magnetic field can then be used 
to either Fourier encode the imaged object, 
to spoil the transverse magnetisation, or even
to sensitize the MR signal to molecular diffusion.

\hfill
    
In practice, the MR scanner's gradient coils have two important properties: their maximum strength and the `rise time'.
The latter is a constraint imposed on the hardware and it is defined as the time it takes for a gradient field to go from 0 to its maximum allowed strength.
Both of these properties are modelled in simulation systems developed by Drobnjak et al. \cite{Drobnjak2006}, Xanthis et al. \cite{Xanthis2014} and Stocker et al. \cite{Stocker2010}.

\hfill

Another property of the gradients is how the amplitude of the field varies through time.
For most clinically available MR imaging sequences, the gradient fields are kept constant in time, or are turned on and off rapidly.
From an imaging point of view, this enables a Cartesian Fourier encoding.
Linear gradients are used in MRI simulators developed by
Benoit-Cattin et al. \cite{Benoit-Cattin2005},
Jochimsen et al. \cite{Jochimsen2004},
Drobnjak et al. \cite{Drobnjak2006},
Overall et al. \cite{Overall2007},
Stocker et al. \cite{Stocker2010},
Liu et al. \cite{Liu2013} and 
Xanthis et al. \cite{Xanthis2014}.
However, in recent years, non-linearly shaped gradients have become popular due to their fast imaging capabilities.
These types of gradient shapes are simulated by 
Jochimsen et al. \cite{Jochimsen2004}, 
Overall et al. \cite{Overall2007} and
Stocker et al. \cite{Stocker2010}.

\hfill

Other properties of the gradient coils are the additional unwanted fields that cause image artifacts.
These include concomitant gradient fields which can appear from the fact that the temporally switching gradient field is accompanied by small terms on the orthogonal components.
These time-varying fields are important in MRI because they cause additional phase accumulation, which must be compensated for to avoid image artifacts.
Stocker et al. \cite{Stocker2010} can model these fields in their simulation system.

\hfill

Finally, gradient fields may suffer from inhomogeneities, the presence of which can result in image distortions.
In terms of simulations, Summers et al. \cite{Summers1986} showed that their presence during the phase encoding step of a sequence can produce errors which will propagate throughout the whole image.
Moreover, Stocker et al. \cite{Stocker2010} showed how unwanted nonlinear gradient fields result in a distorted image.

% % % % % % % % % % % % % % % % % % % % % % % % 
\hfill

\large \textbf{Radiofrequency Coils} \normalsize

The third important hardware component of any MR scanner are the radio frequency coils.
These coils are used to either transmit magnetic fields, receive magnetic fields, or both.
%Roughly speaking, RF coils come in two different flavours: volume and surface coils.
%Volume coils provide a homogeneous field across a large volume and are used for whole-body imaging.
%Volume coils are great for transmitting, but less ideal when used for small regions of interest.
%The reason behind this has to do with their large field of view which will receive not only signal, but also noise.
%On the other hand, surface coils have a high RF sensitivity over a small region of interest.
%Moreover, surface coils have a lower penetration depth than volume coils.
%Surface coils can be coupled in an array to combine the benefits of smaller coils with those of larger ones.

\hfill

Most MRI simulation systems assume homogeneous RF fields across the sample.
However, in practice, RF coils have spatially varying sensitivity patterns.
This phenomenon is simulated by Stocker et al. \cite{Stocker2010}, Xanthis et al. \cite{Xanthis2014} and Hacklander et al. \cite{Hacklander2005} for both transmitting and receiving RF coils.
Additionally, it is often the case that multiple coils are used to either transmit or simultaneously receive signals.
Jochimsen et al \cite{Jochimsen2004}, Liu et al \cite{Liu2014} and Stocker et al \cite{Stocker2010} offer this feature in their MRI simulators.

\hfill

% % % % % % % % % % % % % % % % % % % % % % % % 
% % % % % % % % % % % % % % % % % % % % % % % % 
% % % % % % % % % % % % % % % % % % % % % % % % 
\subsection{Reconstruction}

Image reconstruction is the last component of an MRI experience pipeline.
In conventional approaches where the k-space matrix is fully sampled and populated in a Cartesian way, a simple inverse fast Fourier transform is applied to yield an image.
However, with an increasing need for faster scans, more sophisticated methods such as parallel imaging techniques are now incorporated in clinical MRI scanners.
These methods rely on different types of reconstruction algorithms who generally fall under 3 categories: image domain parallel imaging (PI), such as SENSitivity Encoding (SENSE) type algorithms; k-space domain PI, such as GeneRalized Auto-calibrating Partially Parallel Acquisition (GRAPPA) type algorithms; or hybrid forms such as Array Spatial Sensitivity Encoding Technique (ASSET) and Auto-calibrating Reconstruction for Cartesian Imaging (ARC) \cite{Deshmane2012}.
Among all the MR simulators presented so far, only ODIN \cite{Jochimsen2004} implements a parallel imaging reconstruction algorithm (GRAPPA).

\hfill

Regarding non-cartesian approaches as presented in Section \ref{chapterlabel2sec13}, only MRiLab \cite{Liu2017} reports having a `gridding' module (see Appendix \ref{MRIgridding}).
In general, other simulators make use of dedicated reconstruction tools such as the Berkeley Advanced Reconstruction Toolbox (BART) \cite{Lustig2016} are used for this step.

\clearpage

% % % % % % % % Active MRI Simulators
\section{Active MRI Simulators}
\label{chapterlabel2secMRISIMULATORS}

In this section of the survey we review the current state-of-the-art in MRI simulation systems as freely available software tools that are active today.
Among all the simulators presented so far, 5 are still available and can be downloaded from their dedicated websites.
This review will focus on their most desirable features, while also presenting their limitations.
Table~\ref{table:tableMRISimulators} summarizes our findings, while the following paragraphs delve into the details of each simulator.

\hfill

\begin{table}[h!]
\centering
 \begin{tabular}{||c | c c c c c ||} 
 \hline
 Key Features & \textbf{P} & \textbf{J} & \textbf{O} & \textbf{ML} & \textbf{SB} \\ [0.5ex] 
 \hline\hline
 % % %
 \begin{tabular}{@{}c@{}c@{}c@{}}\textbf{Operating System}\\Linux\\Windows\\Mac OS\end{tabular} & \begin{tabular}{@{}c@{}c@{}c@{}}\\$\checkmark$\\$\checkmark$\\$\checkmark$\end{tabular} & \begin{tabular}{@{}c@{}c@{}c@{}}\\$\checkmark$\\$\checkmark$\\$\checkmark$\end{tabular} & \begin{tabular}{@{}c@{}c@{}c@{}}\\$\checkmark$\\$\checkmark$\\$\checkmark$\end{tabular} & \begin{tabular}{@{}c@{}c@{}c@{}}\\$\checkmark$\\$\checkmark$\\$\checkmark$\end{tabular} & \begin{tabular}{@{}c@{}c@{}c@{}}\\\phantom{-}\\\phantom{-}\\$\checkmark$\end{tabular} \\
 \hline
 % % %
 \begin{tabular}{@{}c@{}c@{}c@{}}\textbf{Programming Language}\\C++\\MATLAB\\Objective-C\end{tabular} & \begin{tabular}{@{}c@{}c@{}c@{}}\phantom{-}\\$\checkmark$\\\phantom{-}\\\phantom{-}\end{tabular} & \begin{tabular}{@{}c@{}c@{}c@{}}\phantom{-}\\$\checkmark$\\$\checkmark$\\\phantom{-}\end{tabular} & \begin{tabular}{@{}c@{}c@{}c@{}}\phantom{-}\\$\checkmark$\\\phantom{-}\\\phantom{-}\end{tabular} & \begin{tabular}{@{}c@{}c@{}c@{}}\phantom{-}\\$\checkmark$\\$\checkmark$\\\phantom{-}\end{tabular} & \begin{tabular}{@{}c@{}c@{}c@{}}\phantom{-}\\\phantom{-}\\\phantom{-}\\$\checkmark$\end{tabular} \\
 \hline
 % % %
 \textbf{Open-Source} & $\checkmark$ & $\checkmark$ & $\checkmark$ & $\checkmark$ & \phantom{-} \\
 \hline
 % % %
 \begin{tabular}{@{}c@{}c@{}}\textbf{Parallelisation}\\CPU\\GPU\end{tabular} & \begin{tabular}{@{}c@{}c@{}}\\$\checkmark$\\\phantom{-}\end{tabular} & \begin{tabular}{@{}c@{}c@{}}\\$\checkmark$\\\phantom{-}\end{tabular} & \begin{tabular}{@{}c@{}c@{}}\\$\checkmark$\\\phantom{-}\end{tabular} & \begin{tabular}{@{}c@{}c@{}}\\$\checkmark$\\$\checkmark$\end{tabular} & \begin{tabular}{@{}c@{}c@{}}\\$\checkmark$\\\phantom{-}\end{tabular}  \\
 \hline
 % % %
 \begin{tabular}{@{}c@{}}\textbf{Sequence Design}\\\textbf{GUI}\end{tabular} & \phantom{-} & $\checkmark$ & $\checkmark$ & $\checkmark$ & $\checkmark$ \\
 \hline
 % % %
 \begin{tabular}{@{}c@{}}\textbf{Coil Design}\\\textbf{GUI}\end{tabular} & \phantom{-} & $\checkmark$ & \phantom{-} & $\checkmark$ & \phantom{-} \\
 \hline
  % % %
 \begin{tabular}{@{}c@{}c@{}}\textbf{Multi-RF}\\Receiving\\Transmitting\end{tabular} & \begin{tabular}{@{}c@{}c@{}}\\\phantom{-}\\\phantom{-}\end{tabular} & 
 \begin{tabular}{@{}c@{}c@{}}\\$\checkmark$\\$\checkmark$\end{tabular} & \begin{tabular}{@{}c@{}c@{}}\\$\checkmark$\\\phantom{-}\end{tabular} & \begin{tabular}{@{}c@{}c@{}}\\$\checkmark$\\$\checkmark$\end{tabular} & \begin{tabular}{@{}c@{}c@{}}\\\phantom{-}\\\phantom{-}\end{tabular}  \\
 \hline
 
 % % %
 \textbf{K-space Visualisation} & \phantom{-} & $\checkmark$ & $\checkmark$ & $\checkmark$ & $\checkmark$ \\
 \hline
 % % %
 \textbf{Motion} & $\checkmark$ & $\checkmark$ & \phantom{-} & $\checkmark$ & $\checkmark$ \\
 \hline
 % % %
 \begin{tabular}{@{}c@{}}\textbf{Multi-pool}\\\textbf{exchange model}\end{tabular} & \phantom{-} & $\checkmark$ & \phantom{-} & $\checkmark$ & \phantom{-} \\
 \hline
 % % %
 \textbf{DW-MRI} & $\checkmark$ & \phantom{-} & \phantom{-} & \phantom{-} & \phantom{-} \\
 \hline
 % % %
 \textbf{fMRI} & $\checkmark$ & \phantom{-} & $\checkmark$ & \phantom{-} & \phantom{-} \\
 \hline
  % % %
 \textbf{Last Update} & Nov'17 & Mar'18 & Nov'16 & Jul'17 & Mar'10 \\
 \hline
\end{tabular}
\caption{Summary of key features of the current active MRI simulators, where \textbf{P} - POSSUM, \textbf{J} - JEMRIS, \textbf{O} - ODIN, \textbf{ML} - MRiLab, \textbf{SB} - SpinBench}
\label{table:tableMRISimulators}
\end{table}

\hfill

% % % POSSUM
\subsection{POSSUM}
\textbf{POSSUM}\footnote{\url{https://fsl.fmrib.ox.ac.uk/fsl/fslwiki/POSSUM}}, or \textit{Physics-Oriented Simulated Scanner for Understanding MRI}, is a software program that creates simulated MRI and fMRI images \cite{Drobnjak2006} \cite{Drobnjak2010}.
POSSUM is part of FSL\footnote{\url{https://fsl.fmrib.ox.ac.uk/fsl/fslwiki/FSL}} (FMRIB's Software Library), a comprehensive library of analysis tools for fMRI, MRI and DTI brain imaging data.
POSSUM is written in C++ and comes equipped with both a graphical user interface and command line tools for the user to interact with the simulator.
The main developers of this project are Ivana Drobnjak and Mark Jenkinson, and the last update made was in November 2017 when Mark Graham released a POSSUM extension capable of producing realistic diffusion-weighted MR datasets \cite{Graham2016}.

\hfill

\subsubsection{Key features and limitations}
The most important features POSSUM has to offer are:
\begin{itemize}
    
    \item It is an open-source software package that can be downloaded and installed on any of the most popular operating systems available (Mac OS, Linux, Windows) as part of the FSL software library: \url{https://fsl.fmrib.ox.ac.uk/fsl/fslwiki/FslInstallation}.
    
    \item It provides a graphical user interface for choosing the input object, setting the sequence parameters ($T_R$, $T_E$, gradient directions, field-of-view, matrix size, etc.), choosing the $B_0$ field strength and inhomogeneities, the motion type, the $T_2^*$ time course and spatial modulation (for fMRI studies) and the signal noise.
    
    \item It uses parallelisation techniques, through the Message Parsing Interface (MPI) programming standard, to speed up the computational process.
    
    \item It can simulate MR signals starting from an input object that can be any segmented anatomical voxel model or any collection of tissue templates. %More specifically, the input object can be a 4D volume where the $4^{th}$ dimension is a tissue type.
    
    \item It is capable of producing realistic simulated MRI, fMRI and DW-MRI datasets.
    
    \item It can simulate spin history effects, motion during readout periods and the interaction with $B_0$ inhomogeneities, by tracking and updating the magnetisation vector through time for every object voxel.
    
    \item It is capable of simulating spatially nonlinear heterogeneous magnetic fields that are also changing in time (except RF excitation period).
    
\end{itemize}

\hfill

Although a powerful and versatile simulator, POSSUM has the following limitations:
\begin{itemize}
    
    \item It uses the hard-pulse approximation, where RF pulses are modelled as instantaneous rotations through an angle as they are considered short enough to not allow for relaxation during this event. 
    
    \item Relaxation is not modelled during the RF pulse.
    
    \item It cannot model the effect of gradient spoiling as an event which forms a distribution of dephased isochromats, thus effectively not allowing for refocusing in subsequent RF pulses.
    
    \item It does not have a dedicated pulse sequence development tool. New pulse sequences have to be created by explicitly programming them.
    
\end{itemize}

\hfill

\subsubsection{Implementation details}

\textbf{Inputs.} POSSUM takes as inputs a variety of image or matrix data.
These data represent different object and scanner related properties and are summarised below.
\begin{itemize}
    
    \item \textbf{Anatomical Object}: The object is a collection of rectangular volume elements of dimensions $(L_x, L_y, L_z)$.
    These object voxels consist of a mixture of different materials or tissues which are considered to be uniformly distributed across the object voxel.
    
    \item \textbf{Tissue Specific Parameters}: The tissue specific parameters (longitudinal relaxation time $T_1$, transverse relaxation time $T_2^*$ and proton density $\rho$) are specified for each tissue type in the object.
    These values are considered to be constant across every object voxel.
    
    \item \textbf{Motion}: Motion can be specified through a 7 column matrix representing the time (in seconds), translations in x, y and z directions (in metres), and rotations in x, y and z directions about the centre of the volume (in radians).
    The motion values in between the two consecutive time points are interpolated.
    
    \item \textbf{Pulse Sequence}: The pulse sequence can be specified through a 8 column matrix representing time (in seconds), RF pulse properties such as: RF angle (in radians), RF frequency bandwidth (in Hz) and RF centre frequency (in Hz), signal readout (binary value), and gradients in x, y and z directions (in T/m).
    The pulse sequence can also be created through a dedicated tool which is part of the POSSUM framework called \texttt{pulse}.
    Two types of sequences are available: an EPI sequence and a GE sequence, for which the user needs to specify the sequence specific properties (RF pulse angle, echo time, repetition time, k-space matrix size and image resolution).
    
    \item \textbf{$B_0$ inhomogeneities}: Inhomogeneities in the main magnetic field can be included in the simulation by providing a set of 3D images representing the $B_0$ inhomogeneities (in Tesla or ppm).
    POSSUM also offers the possibility to create these images through the use of a dedicated tool called \texttt{b0calc}.
    This tool needs as input a 3D binary image which states where tissue or air exist.
    
    \item \textbf{Activation}: The BOLD signal can be included in the simulation through the use of 4D volumes representing each voxel's time-series of $T_2^*$ changes.
    
    \item \textbf{Slice Profile}: The shape of the RF pulse in frequency space can be fed into the simulation through a 2-column matrix representing the amplitude values for every frequency.
    
\end{itemize}

\hfill

\textbf{Outputs.} The POSSUM MRI simulator can offer the user a variety of outputs.
These are summarised below.
\begin{itemize}
    
    \item \textbf{Signal}: POSSUM provides the raw signal in terms of real and imaginary channels for every requested readout point.
    
    \item \textbf{K-Space}: The k-space matrix (magnitude and phase) can be retrieved by using a dedicated tool called \texttt{signal2image} which requires the raw signal as input and the sequence type used.
    
    \item \textbf{Image}: The final image (magnitude and phase) can be retrieved using the same tool as before (\texttt{signal2image}) on the raw signal.
    
\end{itemize}

\hfill

\textbf{Solver.} At the heart of POSSUM is a Bloch equation solver which solves the equation for all object voxels and all tissue types.
The total signal at time $t$ is then calculated as a sum over all the magnetisation vectors from all the object voxels \cite{Drobnjak2006}:
\begin{equation}
    S(t) = \sum_{j \in \Lambda} \sum_{\vec{r}_0 \in \Omega} s_j (\vec{r}_0, t)
\end{equation}
where $\Omega$ is the collection of object voxels, $\Lambda$ is the collection of tissues, $\vec{r}_0$ is the centre of the object voxel and $s_j(\vec{r}_0, t)$ is the signal arising from the $j^{th}$ tissue type of the object voxel whose centre is $\vec{r}_0$.
The signal $s_j(\vec{r}_0, t)$ is calculated analytically and it depends on the proportion of the tissue type, the dimensions of the object voxel, the magnitude of the transverse magnetisation immediately after the last RF pulse, the transverse relaxation effects, gradient induced dephasing, $B_0$ inhomogeneities and motion.
A detailed explanation can be found in \cite{Drobnjak2006}.

% In the rotating reference frame the closed form solutions during relaxation or signal acquisition are:
% \begin{flalign*}
%     M_z(\vec{r}, t) & = e^{-(t-t_0)/T_1} M_z(\vec{r},t_0) + (1 - e^{-(t-t_0)/T_1}) M_0 \\
%     M_{xy}(\vec{r}, t) & = e^{-(t-t_0)/T_2^*} \lvert M_{xy}(\vec{r},t_0) \rvert e^{- i \phi(\vec{r},t)}
% \end{flalign*}

% More specifically, the signal in time for each object voxel is solved analytically.
% \begin{flalign*}
%     S_j(r_0,t) = \, & p(j) L_x L_y L_z \lvert M_{xy}(t_0) \rvert exp\Bigg(- \int_{t_0}^{t} \frac{1}{T_2^*(t)} dt \Bigg) \\
%                  &  sinc(F_x) sinc(F_y) sinc(F_z) \\
%                  &  exp \Bigg( - i \gamma \int_{t_0}^{t} \Big( \vec{G}(t) \cdot (R(t) \vec{r}_0 + T(t) ) + \widetilde{B}_p (\vec{r}_0,t) \Big) dt \Bigg)
% \end{flalign*}
% where 
% $j$ is the $j^{th}$ tissue type, 
% $p(j)$ is the proportion of the $j^{th}$ tissue type present in the object voxel,
% $r_0$ is the centre of the object voxel, 
% $(L_x, L_y, L_z)$ are the voxel dimensions,
% $t_0$ is the time of the last RF pulse,
% $M_{xy}(t_0)$ is the transverse magnetisation immediately after the last RF pulse,
% $F_x = \frac{1}{2} \gamma L_x \int_{t_0}^t (\vec{G}(t) R_x(t) + \widetilde{G}^p_x(\vec{r}_0,t))dt$,
% $\vec{G}(t)$ is the gradient applied at time $t$

\hfill

\textbf{Details of features.}
\begin{itemize}
    
    \item \textbf{Motion}: In POSSUM, motion is modelled as a change of position in the scanner's hardware. 
    More specifically, the object remains static, while the scanner moves relative to it.
    This allows for the closed form solutions of the Bloch equation to be used.
    More details are found in \cite{Drobnjak2006}.
    
    \item \textbf{DW-MRI}: POSSUM is able to produce realistic DW-MRI data sets by introducing diffusion weighting, eddy current-induced gradients and utilizing a spin echo sequence.
    This extension and its details are found in \cite{Graham2016}.
    
    \item \textbf{fMRI}: POSSUM can simulate fMRI data sets by introducing $T_2^*$ variation for every object voxel separately.
    More details are found in \cite{Drobnjak2006}.

\end{itemize}

\hfill

% % % JEMRIS
\subsection{JEMRIS}
\textbf{JEMRIS}\footnote{\url{http://www.jemris.org/index.html}}, or \textit{Juelich Extensible MRI Simulator}, is a versatile, open-source, multi-purpose MRI simulator which can be used for both educational and research purposes.
JEMRIS is a object-oriented C++ software tool developed to achieve performance as well as extensibility.
It relies on four external libraries: \textit{CVODE} - a variable time-stepping integrator for ordinary differential equations and is used for numerically solving the Bloch equations, \textit{MPI} - to parallelise the problem, \textit{Xerces} - to parse inputs written in the general-purpose markup language XML and \textit{GiNaC} - to perform symbolic calculations. 
It provides three MATLAB based graphical user interfaces for creating pulse sequences, for designing the layout of both transmit and receive coil arrays and for executing an MRI simulation.

\hfill

The main developers for this project are Tony Stocker, the project founder, Daniel Pflugfelder and Kaveh Vahedipour.
JEMRIS was last updated in March 2018 when simulation of flow was added.
Most notably, JEMRIS was used to assess the accuracy of carotid plaques MRI \cite{Nieuwstadt2014} and to simulated advanced angiography sequences \cite{Fortin2016}.

\hfill

% % % 
\subsubsection{Key features and limitations}
The most important aspects of JEMRIS are:
\begin{itemize}
    
    \item It is open-source and can be installed on either Mac OS, Linux or Windows: \url{http://www.jemris.org/ug_downl.html}.
    
    \item It provides an easy to use MATLAB GUI for creating arbitrarily complex MRI sequences.
    
    \item It allows for user-defined analytical forms for both gradient shapes and RF pulse shapes. 
    
    \item It comes equipped with the possibility to visually inspect the k-space trajectory.
    
    \item It is capable of producing realistic simulated MRI datasets and single voxel diffusion experiments.
    
    \item It allows the user to design and simulate experiments involving multi-transmit or multi-receive coils.
    
    \item It integrates an open-source framework called \textit{Pulseq} \cite{Layton2017} which allows the user to export a sequence created in JEMRIS to a real MR scanner.
    
    \item It can simulate flow in arbitrary complex geometries \cite{Fortin2016}.
    
\end{itemize}

\hfill

Although a powerful MRI simulator, JEMRIS is not without limitations:
\begin{itemize}
    
    \item It cannot simulate function (such as fMRI), perfusion or diffusion-weighted images.
    
    \item It can become slow when simulating large spin systems or highly discretised input objects. 
    
\end{itemize}

\hfill

% % % 
\subsubsection{Implementation details}

\textbf{Inputs.} The inputs to a JEMRIS simulation are summarised below.

\begin{itemize}
    
    \item \textbf{Anatomical Object \& Tissue Specific Parameters}: JEMRIS incorporates both the anatomical object and the tissue specific parameters into one single input.
    This takes the form of a \textit{HDF5}\footnote{\url{https://support.hdfgroup.org/HDF5/whatishdf5.html}} structure consisting of five 3D matrices (proton density $\rho$, longitudinal relaxation time $T_1$, transverse relaxation times $T_2$ and $T_2^*$ and inhomogeneities in the main magnetic field $\Delta B$) and one scalar value representing the resolution of the object.
    If not provided, $T_2^*$ can also be simulated by adding small off-resonance frequency values to the simulated isochromats sampled from a Lorentzian distribution. 

    \item \textbf{Motion}: Bulk motion is specified similarly to POSSUM (see above), where a 7 column matrix describes, for a given time point (in milliseconds), the x, y and z translations (in millimetres) and the x, y and z rotations (in degrees). 
    Values between two consecutive time points are interpolated. 
    Single voxel \textbf{diffusion} simulations can also be performed in JEMRIS. 
    
    \item \textbf{Pulse Sequence}: In JEMRIS the pulse sequence can either be created through a dedicated GUI or directly by describing its components in an XML file.
    The sequence is represented as a left-right ordered tree structure, where nodes represent loops and leaves represent pulses.
    This allows for a high flexibility in designing the sequence.
    Moreover, as JEMRIS uses the GiNaC library for symbolic calculations, pulses can be specified through arbitrarily complex user-defined equations.
    Thus, a spiral gradient or a Gaussian-shaped RF pulse can be specified analytically.
    The XML file describing the pulse sequence is then parsed by the software through the C++ Xerces library and converted into a C++ sequence tree object.
    In simulations, this C++ object is traversed in order and each leaf is played out.
    
    \item \textbf{Radiofrequency Coils}: RF-receive or RF-transmit coils sensitivity maps can be specified in a JEMRIS simulation. 
    The sensitivity maps can be created through the Coil configuration GUI where one can either choose a Biot-Savart loop or analytically specify the sensitivity profile.
    When a JEMRIS simulation is run for an array of receive coils, signals will be generated for each coil separately.
    For an array of transmit coils, in the simulation the user needs to specify for each RF pulse which channel of the array it is going to be used.
    
    \item \textbf{Gradient Coils}: Non-linear gradients can be simulated in JEMRIS by analytically specifying their spatial dependence.
    Eddy currents can also be included in a simulation through analytically specifying the gradients' temporal dependence.
    Off-resonance effects from concomitant fields can also be simulated in JEMRIS.
    
\end{itemize}


\hfill

\textbf{Outputs.} The outputs produced by a JEMRIS simulation are:
\begin{itemize}
    
    \item \textbf{Signal}: JEMRIS provides the x,y,z values for the net magnetisation at each readout point. 
    
    \item \textbf{K-Space}: The k-space matrix (magnitude and phase) can be retrieved through the dedicated simulation GUI.
    
    \item \textbf{Image}: The final image (magnitude and phase) can be retrieved using the same GUI as for the k-space.
    
\end{itemize}

\hfill

\textbf{Solver.} The JEMRIS simulator user the Bloch equation 
%individually at every position in the input object 
in cylindrical coordinates $(M_r, \phi, M_z)$:

\begin{flalign*}
    \frac{d}{dt} \begin{pmatrix} M_r \\ 
    \phi \\
    M_z \end{pmatrix} = & \begin{pmatrix} cos \phi & sin \phi & 0 \\ 
    - \frac{sin \phi}{M_r} & \frac{cos \phi}{M_r} & 0 \\
    0 & 0 & 1 \end{pmatrix} \\
    & \cdot \begin{bmatrix} \begin{pmatrix} -1/T_2 & \gamma B_z & - \gamma B_y \\
    -\gamma B_z & -1/T_2 & \gamma B_z \\
    \gamma B_y & -\gamma B_x & -1/T_1 \end{pmatrix} \cdot 
    \begin{pmatrix} M_r \, cos \phi \\
    M_r \, sin \phi \\
    M_z
    \end{pmatrix} + 
    \begin{pmatrix} 0 \\
    0 \\
    M_0/T_1 \end{pmatrix}
    \end{bmatrix}
\end{flalign*}

The Bloch equation is solved for every position in the input object using the CVODE library, a variable time-stepping integrator for ODEs \cite{Stocker2010}.
The signal is then calculated for each receiver coil $n$ by integrating over the coil volume $V$:
\begin{flalign*}
    S_n(t) \propto \int_V d^3 r C_n(\vec{r}) M_r(\vec{r}, t) e^{i \phi(\vec{r}, t)}
\end{flalign*}
where $C_n(\vec{r})$ represents the sensitivity map of the $n^{th}$ receiver.

%The magnetic field in the rotating frame is modelled as a sum between the applied gradient fields ($\vec{G}$), arbitrary non-linear gradient fields ($\B_{NLG}$), off-resonance fields ($\Delta B_0$) and the $B_1$ fields of the RF transmit coils.

\hfill

\textbf{Details of features.}
\begin{itemize}
    
    \item \textbf{Motion}: In JEMRIS, motion is modelled as a change of position in the input object.
    Translations and rotations are specified for different time points during the sequence, while the position between two consecutive time points is interpolated. 
    
    \item \textbf{K-space visualisation}: The JEMRIS sequence developer GUI offers the possibility to visualize and inspect the k-space trajectory of the readout and to plot the gradient moments.

\end{itemize}

\hfill

% % % ODIN
\subsection{ODIN}
\textbf{ODIN}\footnote{\url{http://od1n.sourceforge.net/}}, or \textit{Object-Oriented Development Interface for NMR}, is a software framework used to develop and simulate MRI experiments.
ODIN is written in C++ and was last updated in November 2016.
ODIN has an object-oriented design, thus being a modular and flexible software tool.
It comes equipped with two main GUIs: \textit{ODIN} - for developing, testing, visualising and simulating MRI sequences, and \textit{Pulsar} - for generating and simulating arbitrarily complex RF pulses.
The main developer for this project is Thies Jochimsen, who is also the project founder.
To date, ODIN has proved to be a valuable MRI simulator as its sequence programming interface has been advantageous in developing functional magnetic resonance imaging (fMRI) applications \cite{Schafer2004}.

\hfill

The most important aspects of ODIN are:
\begin{itemize}

    \item It is an open-source package that can be installed on either Mac OS, Linux or Windows: \url{http://od1n.sourceforge.net/download.html}.
    
    \item Besides allowing the user to create its own pulse sequences, ODIN comes equipped with a comprehensive list of readily available ones, such as: diffusion tensor imaging (DTI), echo planar imaging (EPI), EPI-based Periodically Rotated Overlapping ParallEL Lines with Enhanced Reconstruction (PROPELLER) and even flow-compensated FLASH sequence for Susceptibility Weighted Imaging (SWI).
    
    \item Besides allowing the user to create its own pulse shapes, ODIN comes equipped with a comprehensive list of readily available ones, such as: slice-selective pulses (SINC, Gauss), 
    adiabatic pulses (SECH, WURST) and 2D pulses with various excitation shapes and different trajectories.
    
    \item It allows the user to visualise the k-space trajectory.
    
\end{itemize}

\hfill

However, ODIN also suffers from the following limitations:
\begin{itemize}
    
    \item It cannot simulate diffusion weighted MR datasets.
    
    \item It uses the hard-pulse approximation, where RF pulses are modelled as instantaneous rotations through an angle and shaped RF pulses are considered as sequences of hard pulses.
    
    \item It does not have a dedicated pulse sequence programming GUI. Although ODIN argues that new pulses can be easily created by programming them, a graphical user interface would speed up this process.
    
\end{itemize}

\hfill

% % % 
\subsubsection{Implementation details}

\textbf{Inputs.} The inputs to ODIN are:

\begin{itemize}
    
    \item \textbf{Anatomical Object \& Tissue Specific Parameters}: 
    The input object is a collection of relaxation constants ($T_1$ and $T_2$), frequency offsets ($\omega$) and proton density values.
    Additionally, the input object can contain diffusion coefficients ($D$) for Bloch-Torrey simulations.
    These values are a function of spatial position and are constant across a voxel of dimensions $(L_x, L_y, L_z)$, covering a frequency range of covering a frequency range $L_{\omega}$.
    
    \item \textbf{Pulse Sequence}: In ODIN the pulse sequence can be either selected from a list of predefined sequences or can be programmed using the dedicated object oriented API.
    
    \item \textbf{Radiofrequency Coils}: ODIN accepts as inputs the sensitivity maps of an array of either receive or transmit coils.
    These maps can be generated through a dedicated command line tool that come with the software tool.
    
\end{itemize}


\hfill

\textbf{Outputs.} The outputs produced by ODIN are:
\begin{itemize}
    
    \item \textbf{Signal}: The raw data signal (real and imaginary channels) can be retrieved.
    
    \item \textbf{Image}: The final images  (magnitude  and phase) can be retrieved for every coil used in the simulation.
    Moreover, when multiple receiver coils are used, the user can also choose to reconstruct the final image by combining all channels in the coil array.
    
\end{itemize}

\hfill

\textbf{Solver.} The ODIN MRI simulator uses the solution 
to the Bloch or Bloch-Torrey equation with piecewise constant fields to iteratively calculate the evolution of all isochromats in the input object.
In order to increase simulation efficiency and to simulate self-diffusion, the ODIN solver calculates the evolution of the magnetization vector at a certain position and for a certain frequency, while also calculating the evolution in its immediate vicinity.
This is done by calculating the intra-voxel magnetization gradients by means of the partial derivatives of the magnetization vector with respect to position and frequency.
This allows the extrapolation of the magnetisation vector at different locations, while also reducing the computational steps.

\hfill

\textbf{Details of features.}
\begin{itemize}
    
    \item \textbf{fMRI}: ODIN allows for simulation of fMRI datasets by supporting time varying input objects.
    
    \item \textbf{GRAPPA}: ODIN can reconstruct images using parallel imaging techqniques.
    More specifically, ODIN allows for GRAPPA reconstructions.
    After specifying the acceleration factor, the acquired k-space can be filled in an undersampled fashion, with auto-calibration lines.
    
\end{itemize}

\hfill

% % % MRILAB
\subsection{MRILAB}
\textbf{MRILAB}\footnote{\url{http://mrilab.sourceforge.net/}} is a rapid and versatile numerical MRI simulator. 
The front-end of MRILAB (main console, design and visualisation tools) is written in MATLAB, while the computational kernels are implemented in C++.
MRILAB comes equipped with three graphical user interfaces: a sequence design GUI, a coil design GUI and a magnet design GUI which can be used to describe the main static field's inhomogeneities.
The main developer for this project is Fang Liu who is a medical imaging research scientist at the Department of Radiology, University of Wisconsin-Madison, United States.
MRILAB was last updated in July 2017 when support for a multi-pool exchange tissue model was added.
To date, MRILAB was used to confirm the accuracy of modified cross-relaxation imaging (mCRI) for mapping myelin in neural tissues and to study the limitations of simplified modeling with single-component simulations for gagCEST \cite{Liu2017}.

\hfill

The most important aspects of MRILAB are:
\begin{itemize}

    \item It is an open-source package that can be installed on either Mac OS, Linux or Windows: \url{https://github.com/leoliuf/MRiLab}.
    
    \item It provides a dedicated toolbox to analyze RF pulses, to design the MR sequence and to configure multiple transmitting or receiving coils.
    
    \item It offers both GPU acceleration and multi-threaded CPU parallel computation.
    
    \item It can achieve high simulation accuracy through simulating highly discretised spin evolutions at small time intervals.
    
    \item It is specialised on simulations of imaging sets based on generalised multi-pool exchange models.
    % unlike others who do single-voxel regime
    
\end{itemize}

\hfill

MRILAB has the following limitations:
\begin{itemize}
    
    \item It achieves spoiling of the transverse magnetisation through a zeroing event which is not sufficient to model real experiment where complete spoiling is not desired.
    
    \item It cannot simulate diffusion weighted MRI datasets.
    
    \item The programming model that makes use of the GPU cores used in MRILAB is CUDA 2.0. This means that MRILAB does not benefit from the latest features of CUDA 9.x that could speed up the simulation even further and also achieve higher accuracy.
    
\end{itemize}

\hfill

% % % 
\subsubsection{Implementation details}

\textbf{Inputs.} The inputs to an MRILAB simulation are:

\begin{itemize}
    
    \item \textbf{Anatomical Object \& Tissue Specific Parameters}: The virtual object used as input to an MRILAB simulation is a MATLAB structure consisting of the following properties:
    gyromagnetic ratio (in $rad/s/T$), chemical shift (in $Hz/T$), 
    number of voxels in x, y and z directions (called \textsc{XDim}, \textsc{YDim} and \textsc{ZDim}),
    voxel resolution in x, y and z directions, 
    number of spin species (called \textsc{TypeNum}),
    a 4D matrix (of size: \textsc{YDim $\times$ XDim $\times$ ZDim $\times$ TypeNum}) representing the proton density at each position in the object, 
    a 4D matrix (of the same size as before) representing the $T_1$ relaxation times at each position in the object, 
    a 4D matrix (of the same size as before) representing the $T_2$ relaxation times at each position in the object,
    a 4D matrix (of the same size as before) representing the $T_2^*$ relaxation times at each position in the object and 
    2 optional properties that can be used for computing the specific absorption rate (SAR).
    In addition, MRILAB comes equipped with a number of pre-existing objects for the user to experiment with.
    
    \item \textbf{Motion}: Motion can be included in an MRILAB simulation and it consists of bulk translation and rotations of the input object.
    These can be specified through the dedicated GUI or by writing an XML file where, depending on the type of motion (translation/rotation), certain properties need to be specified. 
    For translations, the following attributes are needed:
    the motion starting time,
    the motion ending time,
    the sampling interval for the motion,
    a vector in 3D space describing the translation direction
    and an equation of translation with respect to time (e.g. `$2*t$', `$t+200e-3$', etc.).
    For rotations, the following attributes are needed:
    the starting time,
    the ending time,
    the motion sampling interval,
    a vector in 3D space describing the rotation axis and
    an equation of rotation with respect to time (e.g. `$2*t$', `$sin(0.1*t)$', etc.).
    
    \item \textbf{Pulse Sequence}: MRILAB offers a few predefined pulse sequences that can be used in simulations.
    These predefined MRI pulse sequences include: 
    a 3D multishot Fast Spin Echo,
    different gradient echo sequences (a 3D GRE with Cartesian, radial or spiral readout, a 3D multishot EPI),
    a 3D inversion recovery sequence with Cartesian readout, 
    a 3D spin echo sequence with Cartesian readout,
    a 3D SPGR sequence with magnetisation transfer saturation and a few others.
    Moreover, MRILAB has a dedicated sequence design GUI for the user to create its own sequences or edit the predefined ones.
    
    \item \textbf{$B_0$ inhomogeneities}: Inhomogeneities in the main magnetic field can be included in an MRILAB simulation.
    This input can be created through a dedicated magnet design GUI offered with MRILAB by choosing from either a linear $\Delta B_0$ field or a Gaussian $\Delta B_0$ field.
    Moreover, the software allows the user to create their own $\Delta B_0$ fields through symbolic equations (e.g. `$2*X.*Y$', `$2*sin(X)$', etc.) or to import a pre-existing $\Delta B_0$ field data.
    
    \item \textbf{Gradient Coils}: MRILAB offers the possibility to construct a gradient field to be used in simulations.
    This field can be created through a gradient design GUI by choosing from either a linear field (where gradient field vectors need to be specified for x, y and z directions) or a symbolic field (where symbolic equations need to be specified in x, y and z directions).
    As before, user gradient field data can also be included in the simulations.
    
    \item \textbf{Radiofrequency Coils}: MRILAB can simulate an experiment with multiple RF-receive or RF-transmit coils.
    The coils sensitivity maps can be created through a dedicated coil design GUI.
    There are 2 types of predefined coils that can be used: a Biot-Savart coil circle and a Biot-Savart coil rectangle.
    These 2 types can be used to generate an array of coils placed at different positions in the 3D space.
    A user specified coil can also be included in simulations by providing a MATLAB structure consisting of the $B_1$ and $E_1$ fields data and the x, y and z position of the coil centre.
    
\end{itemize}

\hfill

\textbf{Outputs.} The outputs produced by MRILAB are:
\begin{itemize}
    
    \item \textbf{Signal}: After a simulation is complete, the acquired MR signal is stored as an array consisting of the x (real) component and the y (imaginary) components.
    
    \item \textbf{K-space}: K-space locations are also stored and they consist of the $k_x$, $k_y$ and $k_z$ location of the trajectory.
    
    \item \textbf{Image}: The final image (real, imaginary, magnitude and phase) can be retrieved after an MRILAB simulation is complete.
    Additionally, MRILAB includes a `gridding' module which can be used to reconstruct signals which were simulated with non-Cartesian readouts.
    
\end{itemize}

\hfill

\textbf{Solver.} The MRILAB software is based on a discrete Bloch-equation solver.
It simulates the isochromats evolution with small discrete time steps by means of rotation and exponential scaling matrices. 
In addition, the numerical Bloch equation solver can include multiple exchanging water and macromolecular pools for simulating more advanced quantitative MRI methods such as magnetisation transfer and chemical exchange saturation-transfer techniques.

\hfill

\textbf{Details of features.}
\begin{itemize}
    
    \item \textbf{Motion}: Motion is modelled in MRILAB as discrete movements (rotations and/or translations) of the input object.
    
    \item \textbf{Multi-pool exchange model}: The MRILAB software can perform MRI simulations based on a generalised multi-pool exchange tissue model. 
    The tissue is represented by several free and bound proton pools undergoing the magnetization exchange.
    The response of a multi-pool spin system to an MRI pulse sequence is then simulated using the finite differential Bloch-McConnell equation in the rotating frame.
    
\end{itemize}

\hfill

% % % SpinBENCH
\subsection{SpinBench}
\textbf{SpinBench}\footnote{\url{http://www.heartvista.com/spinbench/}} is a software tool designed for the rapid prototyping and analysis of MRI experiments.
SpinBench is written in Objective-C, was last updated in March 2010 and can only run on Mac OS X version 10.5 (Leopard) or newer.
SpinBench is mainly useful as a teaching tool for courses covering MRI spin physics.
It is made up of a comprehensive set of graphical user interfaces for developing, testing, visualising and simulating MRI sequences. 
Its main developers are William Overall and John Pauly from the Department of Electrical Engineering at Stanford University, United States.

The most important aspects of SpinBench are:
\begin{itemize}

    \item It comes equipped with many tutorials and examples, thus making it an easy to use software environment for teaching MRI.
    
    \item Everything in SpinBench is done through an intuitive graphical user interface, thus allowing for easy prototyping and analyzing a wide range of MR pulse sequences and experiments.
    
    \item The experiment can be played out and updated in real-time as the user changes the parameters.
    
    \item It can be used to visualise spin dynamics.
    
\end{itemize}

\hfill

However, SpinBench has the following limitations:
\begin{itemize}
    
    \item It only works on a Mac OS platform and its source code is not open-source.
    
    \item It can only run on a personal computer so it therefore does not scale to more realistic simulations which would need a cluster environment or a GPU platform to run on.
    
\end{itemize}

\hfill

% % % 
\subsubsection{Implementation details}

\textbf{Inputs.} The inputs to SpinBench are:

\begin{itemize}
    
    \item \textbf{Anatomical Object \& Tissue Specific Parameters}: The input to a SpinBench simulation can be set up through the software GUI by choosing simple, predefined 1D or 2D objects. 
    For these objects the user can specify the overall size, the $T_1$, $T_2$ relaxation times, the proton density and the off-resonance frequency of the isochromats.
    
    \item \textbf{Motion}: Bulk motion can be simulated by specifying the isochromats start position and its velocity in x, y and z directions.
    
    \item \textbf{Pulse Sequence}: In SpinBench the pulse sequences can be designed through the GUI by linking together RF pulses and gradient waveforms of predefined and customisable shapes.
    Moreover, SpinBench offers the possibility to code your own pulses in Javascript.
    
\end{itemize}

\hfill

\textbf{Outputs.} The outputs produced by SpinBench are:
\begin{itemize}
    
    \item \textbf{Signal}: SpinBench plots the signal (magnitude and phase) or the x,y,z values of the net magnetisation in real time during the simulation.
    
    \item \textbf{Image}: The final image (magnitude and phase) can also be visualised at any time point during the simulation.
    
\end{itemize}

\hfill

\textbf{Solver.} The SpinBench software is based on a multi-threaded Bloch-equation simulator which computes the signal independently for each value in the input object.
SpinBench is not open-source and more details about the Bloch based implementation are not available.

\hfill

% % % % % 
The five simulators presented in this section are the current state-of-the-art in MRI simulation systems.
These simulators are also freely available and can be downloaded from their dedicated websites.
Each one of them has different strengths and weaknesses as each was developed with a certain methodological question in mind.
In summary, 
POSSUM was initially developed as an fMRI simulator (\cite{Drobnjak2006}) and has recently become a DW-MRI simulator (\cite{Graham2016}), 
JEMRIS was focused on providing a modular and easy to use pulse sequence development tool (\cite{Stocker2010}), 
ODIN was created as an object oriented software tool capable of being executed on different measurement devices (\cite{Jochimsen2004}),
MRILAB was created to simulate responses from multi-pool spin models of arbitrary configurations (\cite{Liu2017})
and
SpinBench was created as an easy to use platform for educational purposes (\cite{Overall2007}).
These different reasons for creating an MRI simulation system have also had an influence on their current capabilities (see Table~\ref{table:tableMRISimulators}).
It is therefore our belief that a single generalised framework for an ideal MRI simulator that leverages the advantages of all the simulators available should be created.
This framework will serve as a guideline for the MRI simulation community.


% In this section of the survey we review the current state-of-the-art in MRI simulation systems as open source software tools that are active today.
% Among all the simulators presented so far, 5 are still available and can be downloaded from their dedicated websites.
% This review will focus on their most desirable features, while also presenting their limitations.

% \hfill

% A highly desirable feature for a realistic MRI simulator is the possibility to simulate motion during the MR acquisition process.
% The reason is that movement artefacts are ubiquitous to MRI and come as a result of a complicated interplay between the type of motion, the pulse sequence, the acquisition strategy and the imaged object.
% Motion causes \textit{blurring}, \textit{ghosting}, signal loss and even appearance of undesired strong signals \cite{Zaitsev2015} in the images.
% Out of all active simulators, four of them simulate rigid-body motion during acquisition. 
% These simulators are called POSSUM \cite{Drobnjak2006}, JEMRIS \cite{Stocker2010}, MRiLab \cite{Liu2013} and SpinBench \cite{Overall2007}.
% However, SpinBench \cite{Overall2007} is currently only available for Mac OS systems.

% \hfill

% Another highly desirable feature for an MRI simulator is its user friendliness.
% All available simulators offer a graphical user interface (GUI) for the users to experiment with.
% However, some of these MR simulation systems were developed with a specific methodological question in mind.
% For example, POSSUM \cite{Drobnjak2010} was created as a realistic functional MRI (fMRI) simulator and has been recently upgraded to allow for simulation of diffusion weighted MR datasets \cite{Graham2016}.
% Although very powerful and accurate, this simulator has become limited to a handful of pulse sequences.
% We argue that it is important for an MRI simulator to be highly customisable in an user friendly way.
% This allows the users to experiment with different pulse sequences, create their own input objects and even coil configurations.
% Among available simulators, JEMRIS \cite{Stocker2010} offers a pulse sequence design GUI which is modular enough to allow for any type of sequence to be rapidly implemented, while also allowing the user to analytically define the pulse gradient shapes.

% \hfill

% A third desirable feature is the availability of multi-RF receiving and transmitting coils, as they have become a standard in clinical MRI scanners \cite{Harvey2009}.
% The former, coupled with parallel imaging reconstruction algorithms allow for faster scans, while the latter can help with patient specific homogenization of the $B_1$ field.
% Therefore, these two are desirable features to have in a realistic MRI simulator.
% Out of all the active players, JEMRIS \cite{Stocker2010}, MRiLab \cite{Liu2013} and ODIN \cite{Jochimsen2004} provide multicoil acquisitions. 
% Additionally, JEMRIS \cite{Stocker2010} and MRiLab \cite{Liu2013} offer the possibility to graphically construct the coil configuration or to load a user defined one.

% \hfill

% Finally, MRI simulations are computationally expensive.
% It is therefore important to allow for parallel implementations that can run either on a personal computer or on a cluster system.
% However, parallel approaches are either limited to the hardware specifications or to having an available cluster system.
% A different approach which allows for very fast simulations is through massively parallel computer hardware such as the Graphics Processing Units (GPUs).
% The current MRI simulator which has this feature is MRiLab \cite{Liu2013}.

% \hfill

% These five MRI simulators are currently available to be downloaded and used.
% However, as mentioned above, all of these exhibit limitations which can either inhibit an accurate representation of a real MRI scan or are too narrowly focused on a single methodological question without the possibility to be extended to other applications.
% To overcome these limitations we propose the introduction of a generalised framework for an ideal MRI simulator that brings together the advantages of all the simulators available.
% In the following section we will therefore present this framework.

% %application programming interface (API) that brings together the advantages of all the simulators available.
\clearpage

% % % % % % % % API Ideal MRI Simulator
%\input{Chapter2/Chapter2P3API.tex}

% % % % % % % % Current MRF Simulation Systems
% % % % % % % % % % % % % % % % 
\section{Magnetic Resonance Fingerprinting Simulation Systems}
\label{chapterlabel2sec2}

All Magnetic Resonance Fingerprinting studies rely on a simulation step in order to create the dictionary of possible signal evolutions.
These simulations require different algorithms based on the type of sequence used.
These sequences are: \ac{bssfp}, \ac{fisp}, and, more recently, magnetisation transfer.

\hfill

To model the behaviour of a magnetisation vector with known relaxation times and proton density values in \ac{bssfp}-type sequences, single isochromat Bloch equations can be used. 
For \ac{fisp}-type sequences, an alternative approach is needed because of the dephasing gradient in each $T_R$.
This algorithm is called the \ac{epg} formalism and it relies on representing a spin system as a discrete set of phase states \cite{Weigel2015}.
More recently, Malik et al. \cite{Malik2017} extended the \ac{epg} framework to model systems with chemical exchange or magnetization transfer. They coined this framework as `EPG-X'.

\hfill

For image-space simulations a similar method is used, where the signal evolutions are generated on a voxel-wise basis and the sampling scheme is prospectively applied.
For example, Pierre et al. \cite{Pierre2016} simulate fully sampled k-space measurements and then apply the nonuniform fast Fourier transform on the corresponding variable density spiral trajectory to create images. 
These methods do not simulate the process of MR acquisition.

\hfill

Simulation of image artifacts in the final quantitative maps have also been a topic of interest in the MRF literature.
%In addition, the final quantitative maps can suffer from a range of artifacts.
One type of artifact that is ubiquitous in MRI, as well as MRF, is motion.
To investigate how motion could impact the reconstructions, Mehta et al. \cite{Mehta2017} acquired motion free data and then retrospectively corrupted it with different types of motion.
Their study was focused on developing a reconstruction algorithm for decreasing the motion sensitivity of MRF.
For this reason, the rigid motion simulations were performed without simulating the process of MR acquisition.

\hfill

On a similar note, Rieger et al. \cite{Rieger2017} performed both a motion simulation study and an \textit{in vivo} motion study.
In their work, they simulated Shepp-Logan phantoms using Bloch-equations in two scenarios: the original MRF-\ac{bssfp} sequence \cite{Ma2013} and their own technique called MRF-EPI \cite{Rieger2016}. 
The motion traces they used were in-plane rotations and translations.
For the \textit{in vivo} study, the volunteer was asked to perform random movements.
Both the \textit{in vivo} study and the simulations show that motion leads to blurring and ghosting artifacts for both methods.
Similar to the previous study, this work also did not simulate the process of MR acquisition.


