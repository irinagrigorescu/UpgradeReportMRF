% % % % % % % % % % % % % % % % 
\section{Magnetic Resonance Fingerprinting Simulation Systems}
\label{chapterlabel2sec2}

All Magnetic Resonance Fingerprinting studies rely on a simulation step in order to create the dictionary of possible signal evolutions.
These simulations require different algorithms based on the type of sequence used.
These sequences are: \ac{bssfp}, \ac{fisp}, and, more recently, magnetisation transfer.

\hfill

To model the behaviour of a magnetisation vector with known relaxation times and proton density values in \ac{bssfp}-type sequences, single isochromat Bloch equations can be used. 
For \ac{fisp}-type sequences, an alternative approach is needed because of the dephasing gradient in each $T_R$.
This algorithm is called the \ac{epg} formalism and it relies on representing a spin system as a discrete set of phase states \cite{Weigel2015}.
More recently, Malik et al. \cite{Malik2017} extended the \ac{epg} framework to model systems with chemical exchange or magnetization transfer (MT). They coined this framework as `EPG-X'.

\hfill

For image-space simulations a similar method is used, where the signal evolutions are generated on a pixel-wise basis and the sampling scheme is prospectively applied.
For example, Pierre et al. \cite{Pierre2016} simulate fully sampled k-space measurements and then apply the nonuniform fast Fourier transform on the corresponding variable density spiral trajectory to create images. 
These methods do not simulate the process of MR acquisition.

\hfill

In addition, the final quantitative maps can suffer from a range of artifacts.
One type of artifact that is ubiquitous in MRI is motion.
To investigate how motion could impact the reconstructions, Mehta et al. \cite{Mehta2017} acquired motion free data and then retrospectively corrupted it with different types of motion.
Their study was focused on developing a reconstruction algorithm for decreasing the motion sensitivity of MRF.
For this reason, the rigid motion simulations were performed without simulating the process of MR acquisition.

\hfill

On a similar note, Rieger et al. \cite{Rieger2017} performed both a motion simulation study and an \textit{in vivo} motion study.
In their work, they simulated Shepp-Logan phantoms using Bloch-equations in two scenarios: the original MRF-\ac{bssfp} sequence \cite{Ma2013} and their own technique called MRF-EPI \cite{Rieger2016}. 
The motion traces they used were in-plane rotations and translations.
For the \textit{in vivo} study, the volunteer was asked to perform random movements.
Both the \textit{in vivo} study and the simulations show that motion leads to blurring and ghosting artifacts for both methods.
Similar to the previous study, this work also did not simulate the process of MR acquisition.

